\chapter{Tecnologias Digitais e seu uso no Ensino de Matemática}
\label{cap:recComputacionais}

% Uso de recursos computacionais, especialmente web, no ensino de Matemática:
% Uma discussão inicial sobre a importância de recursos em software para o ensino de Matemática [vou procurar algumas referências sobre o tema; o seu trabalho de PPE fala um pouco (um parágrafo) sobre isso, não?]\\

% -> a tecnologia é uma, mas seu usos são diversos, criando aí uma dicotomia maleira (lei, zhao). Se bem usada pode [1] aumentar a eficiência das coisas que a gente já faz ou [2] criar novas tecnologias nos permitem fazer coisas novas
% -> em [1]: podem facilitar o estudante a 'enxergar' taxas e gráficos: "O difícil mesmo é encontrar a função!"
    % -> o meio profissional hoje em dia depende muito do domínio de tecnologias, falar do valor que a familiarização com isso pode ter na formação do aluno
    % -> figuras geométricas em livros didáticos e desenhos livres em quadros contém erros conceituais, que podem ser facilmente corrigidos por softwares
% -> Em [2] encaixar web based learning em algum lugar aí
    % -> falar do EaD no covid, né. Se conseguir fonte boa pra isso

Chamamos de \textit{tecnologia} artefatos, produtos, e ferramentas que possuem a capacidade de resolver problemas. Uma tecnologia, ainda que evolua com o tempo, está sempre, a cada momento, em um estado estático. Um celular, por exemplo, é apenas um conjunto de peças que formam um aparelho que você usa no dia a dia, mesmo que uma destas peças seja substituída por uma melhor, durante seu uso, a configuração permanecerá estática. A mesma observação vale para softwares. Uma versão de um sistema operacional, como o Android, mesmo que seja constantemente atualizada, sempre está em alguma versão estática, que pode, inclusive, ser consultada nas configurações do sistema.

O termo \textit{uso da tecnologia} refere-se à aplicação da tecnologia existente para resolver um problema específico. Está em constante evolução, diferentemente da tecnologia em si. Usuários podem encontrar novos contextos aos quais uma tecnologia existente pode ser relacionada e resolver problemas destes novos contextos, além do uso inicial previsto, dando assim novos significados à tecnologia. Por outro lado, mudanças na tecnologia implicam mudanças no uso e nos contextos de uso. Há, assim, uma interconexão e influência mútua entre tecnologia e uso.
\\

Essa evolução no uso da tecnologia pode ter como consequência o aumento na eficiência e qualidade de um artefato qualquer, e pode acabar gerando outros estados estáticos da mesma tecnologia, ou mesmo tecnologias ainda mais novas. Um exemplo didático é o de um homem do período Idade da Pedra Lascada ou Pedra Polida, que utilizava uma pedra pra fazer uma ferramenta, e então usava a sua nova ferramenta pra fazer uma ferramenta mais afiada, e assim por diante.
\\


O escritor inglês Douglas Adams se tornou mundialmente conhecido por suas obras que, além de terrivelmente divertidas, imprimem uma notável carga de viés científico, como pode ser visto nas fantásticas ideias de bugigangas futurísticas, e nas piadas de cunho técnico presentes em O Guia do Mochileiro das Galáxias~\cite{adams2004guia}, ou no documentário Last Chance to See~\cite{adams2013last}, onde os autores, o biólogo Mark Cawardine e o escritor Douglas Adams, viajam o mundo em busca de espécimes de animais em extinção, documentada por Adams em seu estilo particular. Convidado a dar uma palestra no prestigioso evento \textit{Professional Developers Conference}, da \textit{Microsoft}, em 1996, Adams exemplificou muito bem a evolução do uso da tecnologia digital, ao dar sua visão sobre nossa percepção do que os computadores são, corroborando também o fato de que eles --- e as Tecnologias Digitais em geral --- são uma ferramenta de modelagem, e integram diferentes \textit{tecnologias} em si:
\\


% link da palestra na hora +- certa: https://youtu.be/8UNG3cQoOEc?t=1960
% quote começa em 34:01
% link da palestra na hora +- certa: https://youtu.be/8UNG3cQoOEc?t=1960
% quote começa em 34:01
\begin{citacao}
    Por muito tempo, nós não tínhamos muita certeza do que computadores \textit{eram}, eu certamente não tinha. Eu lembro da primeira vez que eu vi um computador, era um Commodore PET [...], em uma loja de equipamentos eletrônicos em Londres. Eu entrei para dar uma olhada e pensei ``Isso é um \textit{negócio} muito legal, é um \textit{troço} maravilhoso". 
    
    Eu fiquei fascinado por ele [o computador], mas eu não conseguia imaginar qual propósito daquilo para mim, porque eu era um escritor [...]. E o motivo de eu não entender qual era o propósito daquilo para mim é porque eu tinha a ideia errada da sua função, eu achei que fosse uma máquina de \textit{soma}, assim como todo mundo. Todo mundo achava que aquilo era algum tipo de super máquina de soma, e desenvolvia aquilo como uma máquina de soma.
    
    E, depois de um tempo, ficamos tão proficientes em manipular e somar números, que percebemos que podíamos fazer os números significarem outras coisas, por exemplo as letras do alfabeto, e inventamos o código ASCII\footnote{\textit{American Standard Code for Information Interchange}, ou \textit{Código Padrão Americano para o Intercâmbio de Informação}, associa um número binário de 8 \textit{bits} a um caractere. É utilizado na codificação textos.} e então percebemos como estávamos sendo burros, como estávamos com a visão acanhada. Por que pensamos que era uma máquina de somar, quando não era uma máquina de somar, é muito mais maravilhoso: uma máquina de \textit{escrever}! E nós desenvolvemos como uma máquina de escrever com uma lista longa de funcionalidades.
    
    Em seguida, olhamos de novo, agora com uma capacidade de \textit{moer os números} mais rápido e mais sofisticada, então a gente faz cada número representar os elementos em um \textit{display gráfico}, com os \textit{pixels}, e assim nós percebemos que nós estávamos com a visão tão acanhada, isso aqui não é nem uma máquina de soma nem uma máquina de escrever. É uma televisão! Com uma máquina de escrever na frente! [...]
    
    Estamos passando por outra iteração disso agora, com o advento da internet, falando que não é nem máquina de soma, nem máquina de escrever, nem televisão, é um \textit{panfleto!} E estamos desenvolvendo como se fosse um milhão de panfletos!
    
    O ponto-chave aqui pra se ter em mente é que [o computador] não é nenhuma dessas coisas. Não é uma máquina de soma, não é uma máquina de escrever, não é uma televisão, com certeza não é um monte de panfletos. Mas é importante que cada uma dessas coisas que nós já conhecíamos, já sabíamos como usar --- máquinas de soma, máquinas de escrever, televisões, panfletos ---, nós modelamos no computador. Porque o computador não é nenhuma dessas, coisas, mas o fato de que conseguimos modelar essas coisas nele nos diz o que ele realmente é: é um aparelho de modelagem.{color{red}
    ~ \cite[após 34 minutos, tradução própria]{adams1996-1}}
\end{citacao}


\section{Recursos Computacionais no Ensino de Matemática}
No contexto da Educação, chamamos genericamente de \textit{recursos computacionais}, ou \textit{TIC} --- Tecnologia de Informação e Comunicação --- às tecnologias oriundas da computação eletrônica. Como exemplos de uso destas tecnologias, temos plotagem de gráficos e construção de figuras geométricas em softwares como GeoGebra. É possível fazer estas construções de maneira dinâmica, e sem os erros naturais de uma construção com régua e compasso físico, ou até mesmo à mão livre. Dessa maneira, o uso dos recursos computacionais resolve, por exemplo, problemas como a falta de dinamismo e imprecisão das construções feitas manualmente. Além disso, o professor ganha autonomia para fazer seus próprios materiais, sem precisar depender de materiais didáticos pré prontos, que podem conter erro~\cite{humberto-sbm}, resolvendo assim o problema da dependência de materiais disponíveis previamente. 

O uso das TIC tem o potencial de reduzir barreiras ao aprendizado através de sua precisão, rapidez e dinamismo. Construções bem definidas e geometricamente proporcionais construídas dinamicamente na frente do aluno, por exemplo, são ferramentas importante na resolução de um dos problemas mais comuns entre os alunos: a dificuldade de ``enxergar" como as funções são formadas~\cite{rezende2012explorando}.

Mas é um erro pensar que as TIC servem apenas para facilitar e agilizar os processos das aulas clássicas. Enquanto ferramenta de modelagem, o computador traz para sala de aula novas possibilidades como programação, design de sites, além da própria modelagem. Segundo \citeonline{lei2007technology}, as atividades não disponíveis em sala de aula tradicional --- como programação e design ---, estão entre as que mais contribuíram para o aprendizado de um grupo de alunos estudado. 
\\

Um exemplo de atividade possibilitada pelas novas tecnologias é sugerido por~\citeauthor{de2020fases}. Em uma atividade feita com alunos de graduação em Biologia, estes foram convidados a pensar em uma maneira de desenhar um objeto matemático --- a parábola --- utilizando apenas retas~\cite[capítulo 1]{de2020fases}. No decorrer da atividade, os alunos tiveram a ideia de desenhar retas secantes à parábola, e conforme estes avançavam na atividade, os professores introduziam as notações usuais do cálculo: $(a, b)$ para se referir ao ponto de abscissa $a$ e ordenada $b$; $x_0$ para se referir a um ponto de referência da parábola, e $x$ para se referir a uma variável associada; $\Delta x$ para especificar o intervalo entre os pontos onde a reta secante intersecta a parábola.
\\




Tendo em mãos uma ferramenta poderosa de modelagem, devemos considerar novas maneiras de utilizá-la na Educação Matemática. Isto envolve pensar em um ensino que não seja voltado apenas para conteúdos, mas para a experimentação da tecnologia no ensino da Matemática e para a \textit{investigação matemática}, valorizando o raciocínio heurístico, a descoberta Matemática, a visualização Matemática, o diálogo com pares --- incluindo os não-matemáticos ---, etc. Segundo \citeonline{de2020fases}:

\begin{citacao}
    Devemos configurar os ambientes educacionais — presenciais, online e blended para que a argumentação, a visualização e as conjecturas tenham status semelhante ao que a resposta certa tem hoje em uma educação ancorada em testes, e cada vez mais em testes estaduais, nacionais e internacionais] ~\cite[p.64]{de2020fases}
\end{citacao}

\citeonline{bairral2013tic} defende uma visão parecida, sugerindo que o uso das TICs possam auxiliar em uma mudança da elaboração de currículos focados em conteúdos para currículos focados em processos, isto é, que não se baseie em que \textit{conteúdos} um professor deve saber para ensinar matemática, e sim que \textit{processos de pensamento ou raciocínio} as disciplinas deveriam contribuir com o seu desenvolvimento. 

Os tais processos podem ser de dois âmbitos: Os inerentes ao pensamento matemático, como ordenação e composição; ou os que são relacionados a estratégias e formas de raciocinar matematicamente, como heurística e cálculo. São diferentes de competências ou objetivos, mas podem contemplar esses dois conceitos. Processos estão associados, por exemplo, a etapas na resolução de um problema: antes de resolver um problema, a \textit{seleção} dos dados é um processo; após selecionadas as informações úteis, essas devem passar por processos de \textit{ordenação, combinação, transformação}, etc. no tratamento de dados; se o aluno tem um conjunto de informações que precisa apresentar em um trabalho, as informações devem passar por um processo de \textit{representação} para serem melhor assimiladas pelos outros alunos.
\\

Nesse contexto, o projeto CDME oferece a professores e alunos um ambiente de exercícios previamente elaborados, onde eles podem ir diretamente para o foco da atividade, sem se preocupar tanto com detalhes técnicos complicados da criação de construções. Podendo conjecturar, argumentar, trabalhar a visualização matemática, se focar nos processos que permeiam cada módulo livremente.

O conteúdo --- as atividades, questionários, softwares--- do CDME também é proposto de maneira flexível, de maneira que o professor possa usar as atividades em diversos contextos e fazer adaptações aos exercícios sugeridos, baseados na atividade~\cite{cdmebortolossi2016conteudos}.

O uso destes recursos também gera novos paradigmas na educação. O chamado \textit{web-based learning}, que consiste em integrar discussões em fóruns, videoconferências e palestras online, e até ambientes de aprendizado virtual\footnote{um software generalista que combina áreas de discussão, salas de conversas, tarefas, e rastreamento de progresso dos alunos, etc., de maneira online} (VLE, do inglês Virtual Leaning Enviroment) à rotina de estudos do aluno, seja ela remota ou presencial~\cite{mckimm2006abc}.

\section{Uma visão sobre a evolução do uso das tecnologias digitais no Ensino da Matemática}

\citeonline{de2020fases} distinguem quatro fases complementares entre si da evolução da tecnologia digital no Ensino de Matemática. A primeira, iniciada nos anos 1980, quando se primeiramente discutia o uso das TICs, é fundamentalmente caracterizada pelos autores pelo uso do software Logo, que enfatiza relações entre linguagem de programação e pensamento matemático.

A segunda fase tem início na primeira metade dos anos 1990 e é caracterizada pelo início da acessibilidade e popularização do computador pessoal, pela chegada dos primeiros softwares de geometria dinâmica e computação algébrica, e por um incentivo por parte de cursos de formação continuada para a utilização desses softwares em contextos escolares. Essa popularização maior foi possibilitada pelas novas Interfaces Gráficas de Usuário, que não exigiam mais conhecimento em programação de computadores.

A terceira fase tem início nos anos 1999, e pode ser caracterizada pelo início do uso da internet em contextos de educação, que permite um contato maior entre mais professores de diversas partes do país e do mundo, do início da possibilidade da utilização de da comunicação hipertextual em contextos de educação, e de novos paradigmas para a Educação a Distância.

A quarta fase, a mais atual, é a menos bem definida, mas pode ser caracterizada pelo uso de internet rápida e de mais fácil acesso a todos, pelo uso de tecnologias portáteis, e por uma grande integração e interconexão dessas tecnologias entre si.
\\

O GeoGebra, por exemplo, é um software de Geometria Dinâmica (GD) e Computação Algébrica (CA), e portanto pode ser descrito como um software dentro da segunda fase. O software ocupa, inclusive, uma posição destaque nessa fase devido a seu pioneirismo em relação à integração entre os conceitos de GD e CA, o que tornou o GeoGebra tão popular entre professores no mundo todo. Posteriormente, influenciado por essa popularidade, o programa passou a contar também com o apoio de uma biblioteca online enorme de atividades e construções postadas por usuários de todo o mundo, em diversos idiomas, e sobre diversos conteúdos, o que só foi possível através do uso difundido da internet, inserindo assim o GeoGebra na quarta fase.