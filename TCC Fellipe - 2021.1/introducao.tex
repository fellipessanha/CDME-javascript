\chapter{Introdução}
\label{cap:introducao}

Em 2007, O Projeto Conteúdos Digitais para o Ensino e Aprendizagem da Matemática do Ensino Médio (Projeto CDME) da Universidade Federal Fluminense foi contemplado com 220 mil reais em um edital para desenvolver uma séries de softwares educacionais voltados para o ensino médio. O material foi produzido com bastante com bastante qualidade, porém, com o passar do tempo, as tecnologias utilizadas em diversos conteúdos desse projeto --- applets em Java --- deixaram de ser suportadas por navegadores de Internet modernos, e, por isso, não podem mais ser utilizados. O ponto de partida deste Trabalho de Conclusão de Curso é resgatar esses conteúdos com tecnologias atuais --- HTML5 e JavaScript, de modo a manter conteúdo multimídia digital de qualidade disponível para todos os estudantes brasileiros. 
\\


No desenvolver do projeto, vimos a necessidade de investigar mais a fundo o que entendemos como tecnologia. Questões como o entendimento intuitivo do que é considerado tecnologia, que muda radicalmente com o tempo, ou a distinção entre \textit{tecnologia} e o \textit{uso da tecnologia} são pontos importante a na discussão do uso de tecnologias na educação matemática. A evolução das tecnologias também podem impactar --- positiva ou negativamente --- seus usos, e por isso é importante que o usuário da tecnologia tenha ciência do uso que que faz dela, de modo a conseguir acompanhar a marcha do avanço tenológico. 
\\

Um exemplo bastante interessante, mas que pode passar despercebido, que ilustra bem como o uso da tecnologia aplicado ao Ensino da Matemática pode ser impactado por fatores diversos, foi gerado pela criação de leis que proibiam o uso de canudos de plástico não-biodegradáveis em estabelecimentos comerciais~\cite{leicanudos}. Um efeito colateral da lei foi gerar uma maior dificuldade na obtenção de canudos de plástico, que eram muito populares entre professores de Matemática como ferramenta em aulas de Geometria, onde serviam para montar sólidos geométricos em uma atividade de estudo de Geometria Espacial. Na situação posta a \textit{tecnologia} em questão é o canudo de plástico, o seu \textit{uso} era o de aresta em atividades de montagem de sólidos geométricos, e a \textit{evolução} foi a lei que proibia o uso de canudos.
\\

Um primeiro ponto que pode causar estranhamento é o de tratarmos, no exemplo do parágrafo anterior, um canudo de plástico como tecnologia. É comum associarmos tecnologias apenas às Tecnologias da Informação e da Comunicação ou a dispositivos eletrônicos e/ou mecênicos sofisticados,
mas é importante frisar que, mesmo o canudo de plástico exige uma estrutura enorme para ser fabricado: o material do qual fazemos o canudo, o plástico, requer toda uma indústria para ser processado do petróleo; o processo de moldagem do plástico exige uma máquina extremamente especializada; a logística envolvida em entregar os canudos nas lojas, seguindo as leis do mercado é extremamente complexa. O pensamento de que ``Tecnologia é só o que foi inventado depois do nosso nascimento"(Frase atribuída a Alan Kay, e varia de acordo com a fonte. Ver~\cite[p. 23]{van2010gaming}), leva a um entendimento errado do que é a tecnologia, e buscamos combater essa visão ao longo desse trabalho.
\\


Na situação descrita, os professores também precisaram encontrar alguma nova \textit{tecnologia} que suprisse suas necessidades, e assim diversos professores adaptaram suas atividades para utilizar, no lugar do canudo --- análogo ao nosso applet Java ---, palitos de madeira --- o nosso novo HTML5 com JavaScript. Da mesma maneira, a ``nova" tecnologia dos palitos abriu um novo uso além do esperado originalmente. O professor tinha agora a oportunidade de abrir uma discussão sobre a questão ambiental do uso de canudos de plásticos contra o palito de madeira.
\\

A discussão do uso de tecnologias no ensino da Matemática nos levou também a traçar um breve panorama do papel do uso de Tecnologias da Informação e Comunicação -- abreviadas como  \textit{TIC}s. Atualmente, a noção é de que o uso das TICs, apesar de não trazer um benefício por si só, pode se aliar a outras novas ideias de educação para proporcionar aos alunos uma nova visão sobre a Matemática.
\\

Também buscamos incentivar todos os interessados em começar ou aprofundar o uso de TIC em seu trabalho, visto que todo o trabalho descrito no capítulo~\ref{cap:partetecnica}, e até a formatação do TCC em si, foi feita com pouca experiência nas tecnologias em si --- GeoGebra, HTML, JavaScript, e \LaTeX.
\\




\section{Organização do Texto}

Este documento é dividido em 6 capítulos. No primeiro, é descrita a proposta do Trabalho, especificado qual problema buscamos resolver, e é apresentado um panorama do que se pretende discutir.
\\

No capítulo~\ref{cap:recComputacionais} tratamos da evolução das tecnologias, da distinção entre objeto tecnológico e seu uso, de como o uso de tecnologias digitais é feito no ensino da Matemática, da evolução dessa relação, como esse uso pode beneficiar ou não os alunos, levando em conta diversos pontos de vista sobre como deve ser feito o uso de tecnologia no ensino de Matemática, e como o projeto do CDME se encaixa nesse tema.
\\

Falamos, no capítulo~\ref{cap:historicoWeb}, a respeito da origem e evolução da internet, o principal meio de comunicação digital do mundo moderno. Tratamos também de como a evolução natural das tecnologias pode acabar gerando incompatibilidades que impedem o funcionamento de um determinado recurso, mostrando assim que deve haver um trabalho ativo de desenvolvedores de software, ainda que sejam eles educadores matemáticos, para manter suas páginas atualizadas, e explica o papel deste trabalho frente a essas questões.
\\

Destrinchamos a história do CDME e do Projeto Ótimo e apresentamos um panorama sobre como o tema da otimização é abordado nas escolas, no capítulo~\ref{cap:CDME}.
\\

Por fim, no capítulo~\ref{cap:partetecnica} é feita a descrição do trabalho de atualização do Projeto Ótimo, em um formato passo-a-passo, no intuito de ajudar quem se depare com um problema parecido no futuro. Também é disponibilizado todo o material que geramos durante o processo em um repositório no GitHub.