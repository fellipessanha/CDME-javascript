% link da palestra na hora +- certa: https://youtu.be/8UNG3cQoOEc?t=1960
% quote começa em 34:01
\begin{citacao}
    Por muito tempo, nós não tínhamos muita certeza do que computadores \textit{eram}, eu certamente não tinha. Eu lembro da primeira vez que eu vi um computador, era um Commodore PET [...], em uma loja de equipamentos eletrônicos em Londres. Eu entrei para dar uma olhada e pensei ``Isso é um \textit{negócio} muito legal, é um \textit{troço} maravilhoso". 
    
    Eu fiquei fascinado por ele [o computador], mas eu não conseguia imaginar qual propósito daquilo para mim, porque eu era um escritor [...]. E o motivo de eu não entender qual era o propósito daquilo para mim é porque eu tinha a ideia errada da sua função, eu achei que fosse uma máquina de \textit{soma}, assim como todo mundo. Todo mundo achava que aquilo era algum tipo de super máquina de soma, e desenvolvia aquilo como uma máquina de soma.
    
    E, depois de um tempo, ficamos tão proficientes em manipular e somar números, que percebemos que podíamos fazer os números significarem outras coisas, por exemplo as letras do alfabeto, e inventamos o código ASCII\footnote{\textit{American Standard Code for Information Interchange}, ou \textit{Código Padrão Americano para o Intercâmbio de Informação}, associa um número binário de 8 \textit{bits} a um caractere. É utilizado na codificação textos.} e então percebemos como estávamos sendo burros, como estávamos com a visão acanhada. Por que pensamos que era uma máquina de somar, quando não era uma máquina de somar, é muito mais maravilhoso: uma máquina de \textit{escrever}! E nós desenvolvemos como uma máquina de escrever com uma lista longa de funcionalidades.
    
    Em seguida, olhamos de novo, agora com uma capacidade de \textit{moer os números} mais rápido e mais sofisticada, então a gente faz cada número representar os elementos em um \textit{display gráfico}, com os \textit{pixels}, e assim nós percebemos que nós estávamos com a visão tão acanhada, isso aqui não é nem uma máquina de soma nem uma máquina de escrever. É uma televisão! Com uma máquina de escrever na frente! [...]
    
    Estamos passando por outra iteração disso agora, com o advento da internet, falando que não é nem máquina de soma, nem máquina de escrever, nem televisão, é um \textit{panfleto!} E estamos desenvolvendo como se fosse um milhão de panfletos!
    
    O ponto-chave aqui pra se ter em mente é que [o computador] não é nenhuma dessas coisas. Não é uma máquina de soma, não é uma máquina de escrever, não é uma televisão, com certeza não é um monte de panfletos. Mas é importante que cada uma dessas coisas que nós já conhecíamos, já sabíamos como usar --- máquinas de soma, máquinas de escrever, televisões, panfletos ---, nós modelamos no computador. Porque o computador não é nenhuma dessas, coisas, mas o fato de que conseguimos modelar essas coisas nele nos diz o que ele realmente é: é um aparelho de modelagem.{color{red}
    ~ \cite[após 34 minutos, tradução própria]{adams1996-1}}
\end{citacao}