\chapter{Otimização}
\label{cap:otimizacao}

% Falar um pouco sobre problemas de otimização no Ensino Médio
% as atividades de otimização do CDME, já mostrando as versões que você fez
% Alguns comentários seus  sobre as atividades\\

% \begin{figure}[h!]
%     \centering
%     \includegraphics[width=8cm]{projeto-otimo.jpg}
%     \caption{captcha}
%     \label{fig:otimo}
% \end{figure}

% -> problemas de otimização visam maximizar ou minimizar determinada função (azevedo 2015)
% -> existem diversas aplicações para esse tipo de problema: economia com lucro; logística com gasto; etc. (azevedo 2015)
% -> por isso é relevante ensinar isso na escola. Muitos alunos terão que lidar com essas coisas em algum ponto da vida
% -> daí entra o CDME, focado em ensinar problemas e otimização contínua e limitada de 1 variável. Focando em tarefas além de algebrismo maluco, como lidar com domínio e etc.

    % -> talvez falar de educação focada em processo, que eu vi naquela fonte de PPE
    
De maneira geral, otimização é o processo de maximizar ou minimizar o valor de uma função, através da uma escolha viável de suas variáveis, dadas as condições impostas pela situação em que o problema se insere. Tais problemas são relevantes por serem extremamente comuns em praticamente qualquer área de atuação profissional: um empresário busca maximizar lucros; um engenheiro mecânico busca a maior eficiência na produção; empresas de logística buscam maximizar a eficiência do roteamento de seus veículos.

Além de contextos humanos e profissionais, problemas de otimização também aparecem em contextos naturais: uma bolha de sabão é tal que a sua superfície é a mínima para um dado volume; a luz sempre percorre a distância que minimize a distância percorrida; e um objeto sempre tende ao equilíbrio em um ponto que minimize a sua energia potencial.

A grande relevância e recorrência deste tipo de problemas torna indispensável sua abordagem em sala de aula. Muitos dos alunos terão que \textit{otimizar} alguma tarefa ao longo de suas vidas, e a experiência de reconhecer domínios e contra-domínios, elaborar e modelar uma regra de função, aprender técnicas que facilitem suas resoluções, quando vista em sala de aula, pode acrescentar muito na formação escolar.

Nesse contexto surge o Projeto Ótimo no CDME. A atividade explora os conceitos citados acima, como domínio, imagem, gráfico de função, entre outros, estimulando conexões entre os aspectos algébrico, numérico, geométrico e verbal da função real. O projeto faz isso através de diversas situações-problema com uma variável. Para que o foco permaneça no processo de otimização propriamente dito, todas estas atividades possuem solução facilmente única e identificável como tal dentro do domínio.

Cada situação é abordada com duas ou mais atividades. Uma atividade trata da otimização da situação apresentada, por exemplo, determinar a medida da base que maximiza a área de um triângulo, dado um perímetro fixo. Outra, aborda a modelagem geral de um problema, por exemplo, determinar a lei de formação de um triângulo de perímetro fixo, em função de sua base, seguindo o exemplo anterior. Essas atividade são acompanhados por uma simulação computacional gráfica da situação, que permite que o aluno interaja com o problema e tenha um melhor entendimento de suas particularidades e limitações. As situações também são acompanhadas de formulários de acompanhamento do aluno, que propõem reflexões adicionais sobre a estrutura do problema de otimização naquele contexto.

% Acho que eu vou deixar em aberto aqui pra a gente falar mais sobre educação focada em processo antes de terminar.
