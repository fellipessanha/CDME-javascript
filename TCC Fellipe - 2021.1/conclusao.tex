\chapter{Conclusão}
\label{cap:conclusao}
% -> uso da tecnologia bem-feito pelos autores do cdme, vim pra atualizar a tecnologia
% -> tecnologia precisa de atualização senão dá ruim
% -> não cair na ladainha de ver o computador como uma televisão
% -> trabalho legal de uso da tecnologia correu o risco de se perder, e ainda tem coisa pra consertar lá no cdme

% seção ensino

Como vimos nos capítulos~\ref{cap:recComputacionais}~e~\ref{cap:CDME}, o Projeto Ótimo --- e também o CDME, como um todo --- faz um uso exemplar das tecnologias digitais no Ensino da Matemática, utilizando de seu alto dinamismo e capacidade de modelagem em problemas de otimização, com o objetivo de estimular as conexões entre os aspectos algébrico, numérico, geométrico e verbal de uma função real, trabalhando a capacidade de modelagem matemática, tão importantes na formação matemática moderna.

Respeitando a função do computador enquanto máquina de modelagem, o trabalho abre portas a propostas modernas em Educação Matemática, que privilegiam não mais os conteúdos em si, mas sim o \textit{fazer matemático}, que envolve o processo de heurístico que leva às respostas em um raciocínio matemático próprio.
\\

Dessa forma, a atualização do Projeto Ótimo aqui feito tem o potencial de impactar o aprendizado de alunos de todo o Brasil, por proporcionar uma abordagem diferente para o ensino de problemas de otimização.
\\

% seção extensão:
Tratamos, durante todo o trabalho, da relação entre as tecnologias e seus usos, de como a perda de popularidade de um determinado uso de tecnologia pode acarretar o desuso de uma tecnologia, ou como o avanço na tecnologia pode, eventualmente, criar incompatibilidades que impossibilitem usos anteriores, ainda que relevantes.
% {\color{red}[juntar com parágrafo anterior]}
O caso do Projeto Ótimo se encaixa neste último caso, e infelizmente correu o risco de se perder na incessável marcha da tecnologia moderna. Porém, com algum conhecimento de programação e um trabalho de pesquisa, foi possível recuperar esse conteúdo e disponibilizá-lo novamente para o grande público, mantendo assim o trabalho universitário de qualidade feito para a sociedade nesse projeto de extensão.
\\

% seção pesquisa

O capítulo~\ref{cap:recComputacionais} expõe uma visão sobre tecnologias, sua evolução, faz um apanhado geral sobre a visão do uso de tecnologias no ensino de Matemática, e avalia como essa relação evoluiu com o tempo, em diferentes fases. Mostra que a literatura aponta para um caminho diferente do seguido atualmente por grande parte das escolas brasileiras,propondo que a ênfase do ensino seja mais voltada para um fazer matemático, em que o caminho que leva às respostas é pensado como parte do processo de ensino.
\\

Apresentamos no capítulo~\ref{cap:CDME} um apanhado geral da história do CDME e do Projeto Ótimo, e investigamos mais a fundo a abordagem do Projeto sobre problemas de otimização e os outros pontos importantes Procuramos explicitar a importância de se trabalhar com situações-problema no ensino de Matemática e usar isso como meio para ensinar os conceitos importantes, além da importância de ter um conhecimento básico da linguagem matemática entendida pelo computador no futuro dos alunos, que já vivem em um mundo permeado por TICs.
\\

% {\color{blue}
Vimos nos capítulos~\ref{cap:historicoWeb}~e~\ref{cap:partetecnica} que a manutenção de recursos educacionais online deve ser um trabalho ativo, e mostramos nesse trabalho que, mesmo com experiência limitada com as tecnologias em questão, é possível realizar essas manutenções, e que alguns conhecimentos a mais de programação de propósito geral conseguem inclusive automatizar alguns processos dessa tarefa.
\\

% seção costura
Por fim, o trabalho buscou se basear no tripé universitário ensino-pesquisa-extensão: por se propôr a fazer um trabalho que impactasse diretamente a educação matemática, através do Projeto Ótimo; por fazer um levantamento do histórico do projeto, do uso de tecnologia na educação, e do histórico do uso das tecnologias, sejam elas da informação ou não; e por promover um trabalho prático de revitalização de uma página que contribui de fato com a sociedade, ao promover o ensino de problemas de otimização de uma maneira tão rica.
% }

% -> Assim como a evolução da tecnologia e a evolução do uso da tecnologia se influenciam (você falou sobre isso),  as obsolescências também podem se influenciar. Quando um problema deixa de ser um problema, as tecnologias que eram utilizadas na solução podem cair em desuso, se não fossem utilizadas para mais problemas (um exemplo disso seria interessante). Da mesma forma, uma tecnologia obsoleta pode comprometer um uso de tecnologia, mesmo que o problema ainda seja relevante.

% -> O Projeto Ótimo, cai no segundo caso. Um uso excelente da tecnologia, tratando muito bem de um problema ainda relevante, mas que perdeu a funcionalidade porque a tecnologia caiu em obsolescência. Isso, inclusive, dá muita força e justificativa ao seu trabalho.

% -> Pode falar que o seu TCC buscou fazer um apanhado de visões sobre tecnologia e seu uso no Ensino de Matemática, uma vez que o que o Projeto Ótimo e tudo o que ocorreu com ele é um ótimo (trocadilho aí) exemplo de como tecnologia e uso se misturam. No caso, o primeiro acabou limitando o segundo.

% -> Lá naquele capítulo sobre uso de TICs no Ensino, você poderia incluir um parágrafo dizendo que quem se utiliza de uma tecnologia deve levar em conta, na hora de a utilizar, que ela não é eterna e pode eventualmente cair em desuso. Temos vários exemplos disso, desde artefatos concretos que se baseiam em material que não são mais encontrados no mercado, como recursos digitais baseados em frameworks caducos.